%% History:
%% May 2019 Tomi Männistö, Antti-Pekka Tuovinen proofreading; 30 vs. 40 cr theses, etc.
%% May 2019 Tomi Männistö changed from babelbib to bibtex; Abstract page (and other pages as well) reformatting.
%% January–May 2019 several issues fixed by Niko Mäkitalo; long fields in abstract
%% March 2018 template file extended by Lea Kutvonen to exploit HYthesisML.cls.
%% Feb2018 This template file for the use of HYgraduML.cls was  modified by Veli Mäkinen from HY_fysiikka_LuKtemplate.tex
%% authored by Roope Halonen ja Tomi Vainio in 2017.
%% Some text is also inherited from engl_malli.tex versions by Kutvonen, Erkiö, Mäkelä, Verkamo, Kurhila, and
%% Nykänen, to accompany tktltiki.cls (by Puolakka 2002).


%% Follow comments to support use.

%%%%%%%%%%%%%%%%%%%%%%%%%%%%%%%%%%%%%%%%%%%%%%%%%%%%%%%%%
%% STEP 1: Choose options for MSc / BSc layout and your bibliographic style
%%%%%%%%%%%%%%%%%%%%%%%%%%%%%%%%%%%%%%%%%%%%%%%%%%%%%%%%%

%%  Language: 
%%      finnish, swedish, or english
%%  Pagination (use twoside by default)  
%%      oneside or twoside,
%%  Study programme / kind of report
%%      csm  = pro gradu in new Computer science MSc;
%%      cs = pro gradu in old Computer Science MSc;
%%      tkt = BSc thesis in new curricula;
%%      tktl= BSc thesis in old curricula;
%%  For MSc choose your line or track:
%%      (30 cr thesis, 2020 onwards, Master of Computer Science programme = csm)
%%      software-track-2020 = Software study track
%%      algorithms-track-2020 = Algorithms study track
%%      networking-track-2020 = Networking study track
%%
%%      (30 cr thesis, Master of Computer Science programme = csm)
%%      sw-track-2018 = Software Systems study track
%%      alko-track-2018 = Algorithms study track
%%      nodes-track-2018 = Networking and Services study track
%%
%%      (30 cr thesis, Master of Computer Science programme = csm)
%%      sw-line-2017 =  Software systems subprogramme
%%      alko-line-2017 = Algorithms, Data Analytics and Machine Learning subprogramme
%%      bio-line-2017 = Algorithmic Bioinformatics subprogramme
%%      nodes-line-2017 = Networking and Services subprogramme
%%
%%      (40 cr thesis, = cs)
%%      sw-line = Software Systems specialisation line 
%%      alko-line = Algorithms specialisation line
%%      bio-line = Algorithmic bioinformatics specialisation line
%%      nodes-line = Networking and Services specialisation line

\documentclass[swedish,onesided,censored,tkt]{HYthesisML}


% In theses, open new chapters only at right page.
% For other types of documents, may ask "openany" in document.
\PassOptionsToClass{openright,twoside,a4paper}{report}
%\PassOptionsToClass{openany,twoside,a4paper}{report}

\usepackage{csquotes}
%%%%%%%%%%%%%%%%%%%%%%%%%%%%%%%%%%%%%%%%%%%%%%%%%%%%%%%%%
%% REFERENCES
%% Some notes on bibliography usage and options:
%% natbib -> you can use, e.g., \citep{} or \parencite{} for (Einstein, 1905); with APA \cite -> Einstein, 1905 without ()
%% maxcitenames=2 -> only 2 author names in text citations, if more -> et al. is used
%% maxbibnames=99 as no great need to suppress the biliography list in a thesis
%% for more information see biblatex package documentation, e.g., from https://ctan.org/pkg/biblatex 

%% Reference style: select one 
%% for APA = Harvard style = authoryear -> (Einstein, 1905) use:
\usepackage[style=authoryear,bibstyle=authoryear,backend=biber,natbib=true,maxnames=99,maxcitenames=2,giveninits=true,uniquename=init]{biblatex}
%% for numeric = Vancouver style -> [1] use:
%\usepackage[style=numeric,bibstyle=numeric,backend=biber,natbib=true,maxbibnames=99,giveninits=true,uniquename=init]{biblatex}
%% for alpahbetic -> [Ein05] use:
%\usepackage[style=alphabetic,bibstyle=alphabetic,backend=biber,natbib=true,maxbibnames=99,giveninits=true,uniquename=init]{biblatex}
%

\addbibresource{bibliography.bib}
% in case you want the final delimiter between authors & -> (Einstein & Zweistein, 1905) 
% \renewcommand{\finalnamedelim}{ \& }
% List the authors in the Bibilipgraphy as Lastname F, Familyname G,
\DeclareNameAlias{sortname}{family-given}
% remove the punctuation between author names in Bibliography 
%\renewcommand{\revsdnamepunct}{ }


%% Block of definitions for fonts and packages for picture management.
%% In some systems, the figure packages may not be happy together.
%% Choose the ones you need.

%\usepackage[utf8]{inputenc} % For UTF8 support, in some systems. Use UTF8 when saving your file.

\usepackage{lmodern}         % Font package, again in some systems.
\usepackage{textcomp}        % Package for special symbols
\usepackage[pdftex]{color, graphicx} % For pdf output and jpg/png graphics
\usepackage{epsfig}
\usepackage{subfigure}
\usepackage[pdftex, plainpages=false]{hyperref} % For hyperlinks and pdf metadata
\usepackage{fancyhdr}        % For nicer page headers
\usepackage{tikz}            % For making vector graphics (hard to learn but powerful)
%\usepackage{wrapfig}        % For nice text-wrapping figures (use at own discretion)
\usepackage{amsmath, amssymb} % For better math
\usepackage[ruled,vlined]{algorithm2e}

\singlespacing               %line spacing options; normally use single

\fussy
%\sloppy                      % sloppy and fussy commands can be used to avoid overlong text lines
% if you want to see which lines are too long or have too little stuff, comment out the following lines
% \overfullrule=1mm
% to see more info in the detailed log about under/overfull boxes...
% \showboxbreadth=50 
% \showboxdepth=50



%%%%%%%%%%%%%%%%%%%%%%%%%%%%%%%%%%%%%%%%%%%%%%%%%%%%%%%%%
%% STEP 2:
%%%%%%%%%%%%%%%%%%%%%%%%%%%%%%%%%%%%%%%%%%%%%%%%%%%%%%%%%
%% Set up personal information for the title page and the abstract form.
%% Replace parameters with your information.
\title{Samtidig lokalisering och kartläggning med datorseende}

% TM: Contributors to template editors now listed in the beginning of the file in comments
\author{Sebastian Sergelius}
\date{\today}


% Set supervisors and examiners, use the titles according to the thesis language
% Prof. 
% Dr. or in Finnish toht. or tri or FT, TkT, Ph.D. or in Swedish... 
\supervisors{Doktor Jeremias Berg}
\examiners{Doktor Patrik Floréen}


\keywords{drönare, kartläggning, lokalisering, samtidig lokalisering och kartläggning, datorseende}
\additionalinformation{\translate{\track}}

%% For seminar papers and such, cover page and abstract
%% requires these three basic items of information.
%% Label as needed and remove comment marks.
%%\level{Seminaariraportti}
%%\programme{Tietojenkäsittelytieteen andidaattiohjelma}
%%\subject{Seminaarisarjan nimi}

%% Replace classification terms with the ones that match your work. ACM
%% ACM Digital library provides a taxonomy and a tool for classification
%% in computer science. Use 1-3 paths, and use right arrows between the
%% about three levels in the path; each path requires a new line.

\classification{\protect{\ \\
\  Computer systems organization $\rightarrow$ Embedded and cyber-physical systems\ \\ 
\  $\rightarrow$ Robotics $\rightarrow$ Robotic autonomy \ \\
}}

%% if you want to quote someone special. You can comment this line out and there will be nothing on the document.
%\quoting{Bachelor's degrees make pretty good placemats if you get them laminated.}{Jeph Jacques}


%% OPTIONAL STEP: Set up properties and metadata for the pdf file that pdfLaTeX makes.
%% Your name, work title, and keywords are recommended.
\hypersetup{
    unicode=true,           % to show non-Latin characters in Acrobat’s bookmarks
    pdftoolbar=true,        % show Acrobat’s toolbar?
    pdfmenubar=true,        % show Acrobat’s menu?
    pdffitwindow=false,     % window fit to page when opened
    pdfstartview={FitH},    % fits the width of the page to the window
    pdftitle={},            % title
    pdfauthor={},           % author
    pdfsubject={},          % subject of the document
    pdfcreator={},          % creator of the document
    pdfproducer={pdfLaTeX}, % producer of the document
    pdfkeywords={something} {something else}, % list of keywords for
    pdfnewwindow=true,      % links in new window
    colorlinks=true,        % false: boxed links; true: colored links
    linkcolor=black,        % color of internal links
    citecolor=black,        % color of links to bibliography
    filecolor=magenta,      % color of file links
    urlcolor=cyan           % color of external links
}

%%-----------------------------------------------------------------------------------

\begin{document}

% Generate title page.
\maketitle


%%%%%%%%%%%%%%%%%%%%%%%%%%%%%%%%%%%%%%%%%%%%%%%%%%%%%%%%%
%% STEP 3:
%%%%%%%%%%%%%%%%%%%%%%%%%%%%%%%%%%%%%%%%%%%%%%%%%%%%%%%%%
%% Write your abstract to be positioned here.
%% You can make several abstract pages (if you want it in different languages),
%% but you should also then redefine some of the above parameters in the proper
%% language as well, in between the abstract definitions.

\begin{abstract}

Robotars förmåga att navigera är ett forskningsområde som har gått framåt under de senaste årtionden. En drönare, som är en flygande robot, har använts för att skaffa visuell information av katastrofområden eller övervaka industriolyckor. Att en robot skulle kunna navigera autonomiskt måste den vara medveten om sitt läge, miljö, kursriktning och hastighet. En viktig del i autonomisk navigering för drönaren är att den kan samtidigt kartlägga omgivningen och lokalisera sig själv i omgivningen. 

För att kameror är billiga och är inte beroende på utomstående hjälp för att skaffa information av omgivningen har vision baserad navigering fått mera uppmärksamhet än de traditionella metoder som till exempel satellitnavigation. Syftet med denna avhandling är att undersöka visionsbaserad navigering från drönarens perspektiv med fokus på samtidig kartläggning och lokalisering och hur detta problem har angripits.
\end{abstract}

% \begin{otherlanguage}{english}
% \begin{abstract}
% Use this otherlanguage environment to write your abstract in another language if needed.

% Topics are classified according to the ACM Computing Classification System
% (CCS), see 
% \url{https://www.acm.org/about-acm/class}:
% check command \verb+\classification{}+. A small set of paths (1--3) should be used, starting from any top nodes
% referred to bu the root term CCS leading to the leaf nodes. The elements
% in the path are separated by right arrow, and emphasis of each element individually can be indicated
% by the use of bold face for high importance or italics for intermediate
% level. The combination of individual boldface terms may give the reader
% additional insight. 
% \end{abstract}
% \end{otherlanguage}

% Place ToC
\newpage
\mytableofcontents
\mainmatter

%%%%%%%%%%%%%%%%%%%%%%%%%%%%%%%%%%%%%%%%%%%%%%%%%%%%%%%%%
%% STEP 4: Write the thesis.
%%%%%%%%%%%%%%%%%%%%%%%%%%%%%%%%%%%%%%%%%%%%%%%%%%%%%%%%%
%% Your actual text starts here. You shouldn't mess with the code above the line except
%% to change the parameters. Removing the abstract and ToC commands will mess up stuff.
%%
%% You may wish to include material to avoid browsing the definitions
%% above. Command \include{file} includes the file of name file.tex.
%% As a side effect, subsequent inclusions may force a page break.

% BSc instructions
%\include{bsc_finnish_contents}
%\include{bsc_english_contents}
% MSc instructions
%\include{msc_finnish_contents}
\chapter{Inledning}

En drönare är ett obemannat luftfordon som kan styras med hjälp av vision baserad-, tröghets- eller satellitnavigation \citep{geospatial}. För att robotar eller drönaren skulle kunna navigera autonomiskt måste den vara medveten om sitt läge, omgivning, hastighet, kursriktning samt start- och slutplats. Andra faktorer som borde beaktas för att flyga från start till slutplatsen är hur drönaren kan behandla data från inbyggda sensorer i sig, det vill säga kartlägga miljön och sin egen position i denna miljö samt hur man planerar vägen utan att kollidera med hinder.

För att drönaren är mer versatila än robotar som inte flyger så finns det mera användningsfall för dem. De kan övervaka markföroreningar, industriolyckor eller växternas behov av vatten och näring \citep{crowdsurveillance}. Drönaren har också använts vid katastrofområden, såsom vid Japans jordbävning för att mäta strålningsvärden i Fukushima och få visuell information om katastrofområdet samt räddning av invandrare vid medelhavet. Mest lovande forskning är inom vision baserad navigering med hjälp av datorseende.

Robotnavigering med hjälp av datorseende är ett aktivt forskningsområde \citep{982903}. För att robotar skulle vara verkligen autonomiska måste de kunna kartlägga sin omgivning och lokalisera sig själv i omgivningen, detta kallas för SLAM (Simultaneous Localization and Mapping). SLAM är ett problem som har fått uppmärksamhet av det vetenskapliga samfundet sedan 80-talet och har sätts som ett rön om det löses. Utmaningar i utvecklingen är att konstruera algoritmer som fungerar i allmänhet, såsom i dynamiska och stora miljön \citep{realslamproblem}.

Avhandlingen kommer att vara en översikt om SLAM problemet och vilka metoder det finns för att kartlägga miljön och lokalisera sig själv med hjälp av datorseende, som refereras också som lokalisering och kartläggning med datorseende. Frågor som behandlas är: Vad är det fundamentala problemet med SLAM? Vilka SLAM metoder skulle kunna användas eller har använts med drönaren och vad begränsar användningen av dessa? Varför har vision baserad navigation fått så mycket uppmärksamhet än de traditionella metoderna?

Resten av avhandlingen är strukturerad enligt följande: I kapitel 2 öppnas bakgrund bakom navigering, kartläggning, lokalisering och bildbearbetning som är nödvändiga för SLAM, kapitel 3 innehåller information om samtidig lokalisering och kartläggning och hur SLAM problemet kan grupperas i olika kategorier enligt data som man har.

\chapter{Bakgrund}

\section{Navigering}

Med navigering anser vi att en drönare planerar och utför en rutt från startplats till målet \citep{geospatial}. För att kunna navigera till målet måste den vara medveten om sitt läge, miljö, kursriktning och hastighet. Autonomisk navigering kräver att drönaren kan undvika hinder, planera sin rutt samt ta omvägar vid behov. I visionsbaserad navigering används visuella sensorer för att få bilder som indata. Bilderna kan sedan behandlas med algoritmer för att få en representation av omgivningen och lokalisera drönaren i omgivningen. Visuella sensorer som används är monokulära, stereo, RGB-D och fisheye-kameror samt kombinationer av dessa. Kamerasensorer är billiga att operera, väger lite och med dem kan man uppfatta färg, textur och former. Traditionella navigeringsmetoder såsom satellit, som använder GPS (Global Positioning System) signaler för lokalisering, och tröghetsnavigering (Inerital Navigation System), som använder accelerometer och gyroskop, får man inte visuell information av omgivningen. Med kameror kan man navigera i GPS-förnekade områden för att kameror är passiva, det vill säga de kan inte bli störda av utomstående signaler eller upptäckas av utomstående entitet. Drönaren kan använda laser och ultraljudsensorer för att navigera men dessa kräver mycket energi och väger mer än kameror \citep{6385934}. För att drönaren byggs mindre än förr, och på grund av detta har begränsad bärförmåga samt batterikapacitet, har dessa sensorer uteslutnas. Med hjälp av vision baserad navigering, som använder kameror, kan man få rikligt med information om omgivningen \citep{geospatial}. Visionsbaserad navigering, som är ett aktivt forskningsområde, är den mest lovande metoden inom robotnavigering.

\section{Datorseende och bildbearbetning}

Bildbearbetning i vision baserad navigering menar att man gör utdrag ur bilder i form av former, kanter, linjer, djuphet, rörelse, färg eller mönster \citep{982903}. Metoder för att uppskatta djuphet med datorseende är att räkna binokulära skillnaden eller rörelseparallax \citep{suomimainittu}. Rörelseparallax betyder att objekt rör på sig snabbare då de är närmare observatören och långsammare då de är långt borta. Med binokulär skillnad har man två kameror som är riktade parallelt i samma linje, kamerornas distans från varandra är känd, deras synfält överlappar och av båda kamerornas bilder utdras samma kännetecken. Från skillnaden ur kännetecknets position i båda av kamerornas bilder kan man estimera djupheten. Med en monokulär kamera kan man uppskatta djuphet av bilder baserat på rörelseparallax. För att bygga djuphetskarta med monokulärkamera och rörelseparallax behöver man veta distansen till ett objekt då navigeringen börjar och som indata får man bilder och rörelseinformation av roboten \citep{suomimainittu}. \cite{suomimainittu} undersökte detta och märkte att stereokameror fungerar bättre då objekten är nära medan monokulära kameror med hjälp av rörelseparallax kan uppfatta mer precis distans då distansen växer. 

Man kan också utjämna bilder \citep{mapbuildingsift}. Detta menar att man blurrar bilder för att få bort brus och då man gör utdrag ur bilder i form av egenskaper får man färre, men bättre träffar. För att få former, kanter eller linjer ur bilder finns det olika algoritmer såsom SIFT, HARRIS, SURF, ORB, med mera \citep{8930783, slamproblem, mapbuildingsift}.

\section{Kartläggning av omgivningen}

En karta om miljön kan representeras i två (2D) eller tre (3D) dimensioner \citep{geospatial}. Kartor kan sparas i olika format, såsom datorstödd konstruktion (CAD, Computer-Aided Design), rutnät beläggning (Occupancy Grid) eller topologisk karta \citep{982903}. Kartan kan vara färdigt sparad för drönaren eller så kan miljön kartläggas från bilderna av sensorerna då drönaren flyger \citep{geospatial}. Med 3D volymetriska sensorer kan man konstruera en 3D modell och spara denna information i en Octree struktur. Med strukturen kan data om miljön packas i mindre format utan att tappa möjligheten att uppdatera informationen vid behov. En annan metod som tas upp är med stereovisionsalgoritmer göra en djuphetskarta och behandla data till plana ytor som minskar på missvisning som uppstår med användning av stereovision algoritmer när man bygger upp djuphetskartan.

\section{Lokalisering i omgivningen}

Med att en robot lokaliserar sig menar vi att den tar reda på sin position \citep{982903}. Då man har en uppfattning om miljön, det vill säga en karta, så kan en drönare lokalisera sig själv med att ge den landmärken som den hittar då den navigerar. Från bilderna ur drönarens kameror identifieras landmärken, sedan matchas observerade landmärken med de som finns i kartan och efter det kan man beräkna positionen av roboten. Positionen av roboten kan beräknas också utan kartan med att beräkna distansen till landmärken och distansen mellan landmärken som extraheras ur bilder \citep{realslamproblem}. För att veta sin position i miljön används sannolikhetsberäkning med hjälp av rekursiv Bayes, som betyder att man räknar sannolikheten av robotens position när man vet riktningen, hastigheten och observerade landmärken för varje tidpunkt. Man kan dela lokalisering till global- och lokal lokalisering \citep{982903, globalsubmaps}. Med global lokalisering har roboten ingen vetskap om sin position i början. I lokal lokalisering har roboten en ungefärlig eller exakt vetskap om sin plats vid början av navigeringen som den fått som indata. Lokala lokaliseringsmetoder strävar för att korrigera fel som uppstår av robotens rörelse. Globala lokaliseringstekniker kan återvinna från verkliga misstag då man estimerar robotens position.

\chapter{Samtidig lokalisering och kartläggning}

Samtidig lokalisering och kartläggning (SLAM) är ett av de grundläggande problem i robotnavigering \citep{realslamproblem}. SLAM är ett problem där en autonom robot som inte har tidigare information om sin plats eller omgivning skall samtidigt bygga en karta och lokaliserar sig själv till exempel med hjälp av att identifiera landmärken. Detta kan uppnås med sannolikhetsberäkning baserat på tid där man tar i beaktande riktningen av roboten, distansen roboten rör på sig, landmärken som är invariant för rörelse och observationer som roboten gör vid varje tidpunkt. Några lösningar för SLAM som baserar sig på sannolikheträkning är Extended Kalman Filter (EKF-SLAM) och FastSLAM \citep{realslamproblem}. 

SLAM-problemet har delproblem som borde lösas för att få robotar att navigera autonomiskt \citep{slamproblem}. Det svåraste problemet i flesta SLAM-lösningar är dataföreningsproblemet som betyder att man identifierar två olika landmärken som en och samma. Detta problem kan uppstå redan vid korta rörelsen av robotar eller då en robot har navigerat och kommer till en plats som den har redan varit i förr, detta kallas för loopstängning (Loop Closure). 

För autonomisk navigering behöver en robot veta sitt läge i miljön \citep{geospatial}. Med hjälp av kameror, bildbearbetning och beräkning kan miljön kartläggas, i helhet eller delvis, och drönaren lokalisera sig själv, detta kallas också för visuell lokalisering och kartläggning. Detta kan delas i tre kategorier, som är kartlösa (mapless), kartbaserade (map-based) och kartbyggande (map-building) system. 

\section{Kartlösa system}

I system utan karta navigerar drönaren bara med hjälp av att observera tydliga egenskaper i miljön \citep{982903}. Dessa kan vara till exempel väggar, dörrar, möbler eller andra landmärken. Metoder som används inom kartlösa system är optiskt flöde (Optical flow) och spårning av egenskaper (Feature tracking). 

\subsection{Optiskt flöde}

Santos-Victor et. al har använt optiskt flöde i en robot för att imitera bin \citep{341094}. De placerade två kameror på varsin sida av en robot och beräknade hastigheten från bilderna av båda kamerorna med att ta fem bilder som utjämnas med Gaussisk oskärpa och de två sista utjämnade bilder används för att räkna medeltalet av optiska flödesvektorer. Om flödesvektorerna var samma på båda sidorna så for roboten rakt framåt, annars så far den mot den sida vilkens hastighet är mindre. Metoden som Santos-Victor et al. använde för roboten fungerar bara om båda kamerorna är symmetriskt riktade från varandra när man tar i beaktande rörelseriktningen av roboten \citep{982903}. Denna teknik fungerar dåligt i texturlösa miljö och det går inte att byta rörelseriktningen. Fastän metoden som \cite{341094} använde tog bara i beaktande horisontala flödesvektorer så har sedan detta användning av optiska flödesmetoder forskats mera. Nuförtiden kan man använda drönaren att uppskatta avstånd, hålla sin höjd, undvika hinder, beräkna hastighet och landa på en plattform som rör på sig med hjälp av optiskt flöde \citep{6564752}.

\subsection{Spårning av egenskaper}

Spårning av egenskaper (Feature tracking) används för att skaffa information om objekt, så som linjer, hörn och olika former som är invarianta \citep{geospatial}. Med hjälp av dessa egenskaper av objekten och deras relativa rörelse i sekventiella bilder kan man bestämma robotens position. Då drönaren navigerar i området, så kommer den troligtvis att se dessa objekt från olika perspektiv, som hjälper drönaren att navigera. EKF-SLAM och FastSLAM faller under denna kategorin \citep{8930783}. Latif och Saddik har gjort en experiment inomhus med en drönare där de använde VoSLAM (Visual Odometry SLAM) som delar upp bildbearbetningen och 3D kartbyggande i egen processor och uppdatering av kartan samt lokaliseringen i en annan processor \citep{8930783}.

\section{Kartbaserade system}

Med kartbaserade system har drönaren en färdig vetskap om miljön som kan vara i form av geometriska modeller, topologiska kartor eller förhållande mellan landmärken \citep{982903}. Idén är att då roboten navigerar prövar den hitta landmärken från bilder som är lika till de landmärken som roboten vet om. När den hittat dessa så kan roboten beräkna sin position i miljön. Med hjälp av karta kan drönaren planera sin rörelse i förhand och ta omvägar där det behövs \citep{geospatial}. 

Kartbaserad lokalisering med datorseende kan delas upp i fyra steg, som är att skaffa sensorinformation, upptäcka landmärken från informationen med bildbearbetning, matcha observationerna med förväntan och beräkna positionen \citep{982903}. Det svåraste steget av dessa är att matcha observationerna med förväntan. Detta kallas för dataföreningsproblemet. Man kan inte med full säkerhet veta robotens position och då är det svårt att matcha landmärken från rätt synvinkel.

Då kartan finns kan man fokusera på lokalisering av roboten \citep{982903}. I global lokalisering måste man lita på att man kan förena observationerna med informationen man har och ta i beaktande osäkerheten med att någon av observationerna kan matcha flera av de landmärken man vet om. Detta problem kan lösas till exempel med Monte Carlo lokalisation som i korthet fungerar så att en robot antar med lika sannolikhet sin position i kartan och när den rör på sig så observerar den ny information och kan beräkna sannolikheten att hur de som den observerar matchar den karta som den har \citep{772544}.

I lokal lokalisering måste lokaliseringsalgoritmen hålla koll på osäkerheten av robotens position då den rör på sig och vänder \citep{772544}. Då osäkerhet av sin position är för stor stannar roboten och beräknar sin position med hjälp av visuell information av bilderna, alltså minskar på osäkerheten.

\section{Kartbyggande system}

Kartbyggande system kan användas då det är svårt att navigera med en existerande karta om omgivningen eller om kartan inte finns, som i katastrofområden \citep{geospatial}. Roboten kan vara medveten eller omedveten om sin position vid början av kartbyggandet \citep{globalsubmaps}. Kartbyggande systems bildbearbetning kan delas tre kategorier, som är indirekta, direkta och hybrida metoder, som sammanslår indirekta och direkta metoder \citep{geospatial}.

\subsection{Indirekta metoder}

I bildbearbetning som använder indirekta metoder tar man kännetecken ur bilden som är invariant för rotation, synvinkeländringar och rörelseoskärpa, dessa ges som indata som sedan kan användas för rörelseuppfattning och lokalisering \citep{geospatial}. Ett sätt att konstruera en karta är att beakta robotens rörelse och synvinkel \citep{globalsubmaps}. \cite{mapbuildingsift} har gjort dessa samt användt indirekta metoder i sin artikel \citetitle{mapbuildingsift} för att bygga en 3D-karta \citep{mapbuildingsift}. De tar bilder från stereokameror, utjämnar bilderna och använder SIFT (Scale-Invariant Feature Transform) algoritm för att extrahera egenskaper ur bilderna. Med denna metod har de kunnat konstruera en 3D karta av omgivningen baserat på landmärken utan att spara korrelationsmatris mellan landmärken som minskar häftigt på behov av beräkning. Denna metod har ändå problem då det kommer till loopstängning och uppdatering av kartan \citep{globalsubmaps}. Med att använda samma spårning av egenskaper, kartlägga delar av områden och senare bygga en stor global karta fick forskarna loopstängning och kart-uppdateringen löst i begränsad miljö. I korthet kartlägger de, enligt robotens synvinkel, delar av kartan och från kartorna bredvid varandra kunde de identifiera landmärken och foga ihop kartorna till en stor karta. Indirekta metoder fungerar dåligt i strukturlösa miljön, för det finns inte egenskaper att extrahera ur bilder \citep{geospatial}.

\subsection{Direkta metoder}

Direkta metoder fungerar bättre i strukturlösa miljön \citep{Engel2014LSDSLAMLD}. Som i indirekta metoder där man prövar hitta många små kännetecken ur bilder så i direkta metoder använder man hela bilden för att hitta geometriska egenskaper. Med hjälp av dessa så kan man hitta täta korrespondenser och konstruera en detaljerad karta med extra processorberäkning \citep{geospatial}. 

\chapter{Sammafattning}

\iffalse
Mäst är inomhus av drönaren
Problem med beräkning, pga batterikapacitet och komplexa algoritmer och bildbearbetningsalgoritmer
6DoF globalsubmaps
\fi



%%%%%%%%%%%%%%%%%%%%%%%%%%%%%%%%%%%%%%%%%%%%%%%%%%%%%%%%%
\cleardoublepage                          %fixes the position of bibliography in bookmarks
\phantomsection
\addcontentsline{toc}{chapter}{\bibname}  % This lines adds the bibliography to the ToC
\printbibliography

%%%%%%%%%%%%%%%%%%%%%%%%%%%%%%%%%%%%%%%%%%%%%%%%%%%%%%%%%
\backmatter
\begin{appendices}

%\input{instructions_english}
% \input{instructions_finnish}

% \appendix{Sample Appendix\label{appendix:model}}
% usually starts on its own page, with the name and number of the appendix at the top. 
% The appendices here are just models of the table of contents and the presentation. Each appendix
% Each appendix is paginated separately.

% In addition to complementing the main document, each appendix is also its own, independent entity.
% This means that an appendix cannot be just an image or a piece of programming, but the appendix must explain its contents and meaning.

\end{appendices}
%%%%%%%%%%%%%%%%%%%%%%%%%%%%%%%%%%%%%%%%%%%%%%%%%%%%%%%%%

\end{document}
